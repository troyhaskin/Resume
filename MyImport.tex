\usepackage[USenglish]{babel}        %.................................... Define the language of the typeset.
\usepackage[T1]{fontenc}             %.................................... Modern Western Font Typsetting scheme
\usepackage{lmodern}
\usepackage[margin=0.5in]{geometry} %. Define margins
\usepackage{multirow}                %.................................... Enable multirow command in tables
\usepackage{setspace}
%\usepackage{hyperref}
\usepackage{bookmark}
\usepackage{xstring}
%\usepackage{ifthen}
\usepackage{xifthen}
\usepackage{textcase}
\hypersetup{
    unicode            = true                                             , % non-Latin characters in Acrobat' bookmarks
    pdftoolbar         = true                                             ,     % show Acrobat' toolbar?
    pdfmenubar         = true                                             , % show Acrobat's menu?
    pdffitwindow       = false                                            , % window fit to page when opened
    pdfstartview       = {FitH}                                           , % fits the width of the page to the window
%    pdfpagemode        = UseNone                                          ,
    pdfsubject         = {Resume for  Troy C. Haskin}                     , % subject of the document
    pdfkeywords        = {Resume}                                         , % list of keywords
    pdfnewwindow       = true                                             , % links in new window
    colorlinks         = true                                             , % false: boxed links; true: colored links
    pdftitle           = {Troy Haskin | Resume}                           , % title
    pdfauthor          = {Troy C. Haskin}                                 , % author
    pdfcreator         = {pdfTeX v3.1415926-1.40.14}                      , % creator of the document
    pdfproducer        = {pdfLaTeX, MiKTex 2.9, TeXnicCenter 2.0 Beta 2}  , % producer of the document
    linkcolor          = blue                                             ,
    urlcolor           = blue                                             ,
    pdfdisplaydoctitle = true                                             ,
    bookmarksopen      = false                                            ,
}

\renewcommand{\leftmargini}   {1em}
\renewcommand{\leftmarginii}  {0em}
\renewcommand{\leftmarginiii} {1.5em}
\renewcommand{\leftmarginiv}  {0em}





% =============================================================================================== %
%                                        CSV Creation Commands                                    %
% =============================================================================================== %
\makeatletter

\newcommand{\MakeCommand}[2]{
    \expandafter\newcommand\csname #1\endcsname{#2}
}

\newcommand{\ReMakeCommand}[2]{
    \expandafter\renewcommand\csname #1\endcsname{#2}
}

\newcommand{\CSVAppend}[2]{ % #1 = List name, #2 = Token to push on to right

    \@ifundefined{#1}
        {
            \MakeCommand{#1}{#2}
            \expandafter\newcommand\csname #1Append\endcsname[1]{\CSVAppend{#1}{##1}}
            \expandafter\newcommand\csname #1Prepend\endcsname[1]{\CSVPrepend{#1}{##1}}
        }{
            \expandafter\edef\csname #1\endcsname{\csname #1\endcsname,{#2}}
        }
}

\newcommand{\CSVPrepend}[2]{ % #1 = List name, #2 = Token to push on to left

    \@ifundefined{#1}
        {
            \MakeCommand{#1}{#2}
            \expandafter\newcommand\csname #1Append\endcsname[1]{\CSVAppend{#1}{##1}}
            \expandafter\newcommand\csname #1Prepend\endcsname[1]{\CSVPrepend{#1}{##1}}
        }{
            \expandafter\edef\csname #1\endcsname{{#2},\csname #1\endcsname}
        }
}

\newcommand{\CSVGet}[1]{\csname#1\endcsname}

\newcommand{\CurrentValue}{}
\newcommand{\CurrentName}{}
\def\CSVPopAndUpdateValue#1,#2{
    \renewcommand{\CurrentValue}{#1}
    \CSVMake[overwrite]{\CurrentName}{#2}
}

\newcommand{\CSVPop}[1]{ % #1 = List name, #2 = Token to push on to left

    \@ifundefined{#1}
        {
%            \MakeCommand{#1}{#2}
%            \expandafter\newcommand\csname #1Append\endcsname[1]{\CSVAppend{#1}{##1}}
%            \expandafter\newcommand\csname #1Prepend\endcsname[1]{\CSVPrepend{#1}{##1}}
        }{
            \renewcommand{\CurrentName}{#1}
            \GetTokens{CurrentToken}{RestOfList}{,\CSVGet{#1}}
        }
}







\newcommand{\CSVMake}[3][]{ % #1 = List name, #2 = Token to push on to right

    \@ifundefined{#2}
    {
        \MakeCommand{#2}{#3}
        \expandafter\newcommand\csname #2Append\endcsname[1]{\CSVAppend{#2}{##1}}
        \expandafter\newcommand\csname #2Prepend\endcsname[1]{\CSVPrepend{#2}{##1}}
    }{
        \ifthenelse{\equal{#1}{}}
            {
                \typeout{*** CSV Note: Creation ignored since list already exists. ***}
            }{
                \typeout{*** CSV Note: Overwriting already-created list. ***}
                \ReMakeCommand{#2}{#3}
                \@ifundefined{#2Append}{
                    \expandafter\newcommand\csname #2Append\endcsname[1]{\CSVAppend{#2}{##1}}
                }{
                    \expandafter\renewcommand\csname #2Append\endcsname[1]{\CSVAppend{#2}{##1}}
                }
                \@ifundefined{#2Prepend}{
                    \expandafter\newcommand\csname #2Prepend\endcsname[1]{\CSVAppend{#2}{##1}}
                }{
                    \expandafter\renewcommand\csname #2Prepend\endcsname[1]{\CSVAppend{#2}{##1}}
                }
            }
    }
}


% ===================================================================================== %
%                                     <= and >= Commands                                %
% ===================================================================================== %
\newcommand{\IfGreaterThanEqualTo}[4]{
    \ifnum#1>#2
        #3
    \else
        \ifnum#1=#2
            #3
        \else
            #4
        \fi
    \fi
}

\newcommand{\IfLessThanEqualTo}[4]{
    \ifnum#1<#2
        #3
    \else
        \ifnum#1=#2
            #3
        \else
            #4
        \fi
    \fi
}



% ===================================================================================== %
%                                      For-Loop Definition                              %
% ===================================================================================== %

% Initilaze the recursive command (see usage below) ----------------------
\newcommand{\ForLoopRecursion}{}

% Define the loop counter ------------------------------------------------
\newcounter{ForLoopCounter}
\setcounter{ForLoopCounter}{0}

% Iterator command -------------------------------------------------------
\newcommand{\For}[4][1]{
% Arguments
%   #1 = increment (optional)
%   #2 = start value
%   #3 = end value
%   #4 = <code>

    % Redefine the command used for recursion
    \renewcommand{\ForLoopRecursion}{
        #4                                          % Excute <code>
        \addtocounter{ForLoopCounter}{#1}           % Increment the counter
        \For[#1]{\value{ForLoopCounter}}{#3}{#4}    % Recurse
    }

    % Set the counter to the start value 
    % After the recursion begins, #2 is the current value of the counter and not the start value.
    \setcounter{ForLoopCounter}{#2}
    
    % Switch to deal with positive vs. negative increments (decrements).
    \ifnum #1 > 0                                           % If positive increment
        \IfLessThanEqualTo{\value{ForLoopCounter}}{#3}{     % Execute while the LoopCounter is less than or equal to the end value
            \ForLoopRecursion
        }{}
    \else                                                    % If negative increment
        \IfGreaterThanEqualTo{\value{ForLoopCounter}}{#3}{   % Execute while the LoopCounter is greater than or equal to the end value
            \ForLoopRecursion
        }{}
    \fi
}


% ===================================================================================== %
%                                Array Building Commands                                %
% ===================================================================================== %

\newcommand{\DefineNewCounter}[2]{
    \newcounter{#1}\setcounter{#1}{#2}
}

\DefineNewCounter{ArrayWorkCounter}{0}
\newcommand{\ArrayStart}    {ArrayStart}
\newcommand{\ArrayEnd}      {ArrayEnd}
\newcommand{\ArrayPosition} {ArrayPosition}
\newcommand{\ArrayCount}    {ArrayCount}

\newcommand{\ArrayMake}[1]{% #1 = Array name

    \MakeCommand{#1\ArrayPosition}{#1\ArrayPosition}
    
    \DefineNewCounter{#1\ArrayPosition} {0}
    \DefineNewCounter{#1\ArrayStart}    {1}
    \DefineNewCounter{#1\ArrayEnd}      {0}
    \DefineNewCounter{#1\ArrayCount}    {0}

}

\newcommand{\ArrayClear}[1]{% #1 = Array name
    \For[1]{1}{\value{#1\ArrayEnd}}{
        \ReMakeCommand{#1\roman{ForLoopCounter}}{\relax}
    }
    \ReMakeCommand{#1\ArrayPosition}{\relax}
    \setcounter{#1\ArrayPosition} {0}
    \setcounter{#1\ArrayStart}    {1}
    \setcounter{#1\ArrayEnd}      {0}
    \setcounter{#1\ArrayCount}    {0}
}



\newcommand{\ArrayPush}[2]{% #1 = Array name, #2 value to push

    % increment counter
    \addtocounter{#1\ArrayCount}    {1}
    \addtocounter{#1\ArrayEnd}      {1}

    % Naming scheme for the Array commands
    \protected@edef\PushCommand{#1\roman{#1\ArrayEnd}}
    
    % Making the Array commands
    \@ifundefined{\PushCommand}{
        \MakeCommand{\PushCommand}{#2}
    }{
        \ReMakeCommand{\PushCommand}{#2}
    }
}

\newcommand{\ArrayPop}[1]{% #1 = Array name
    \ifnum\value{#1\ArrayCount}>0
        \csname #1\roman{#1\ArrayEnd}\endcsname
        \addtocounter{#1\ArrayCount}    {-1}
        \addtocounter{#1\ArrayEnd}      {-1}
    \fi
}

\newcommand{\ArrayPopAndStore}[2]{% #1 = Array name, #2 name of command to store the popped value
    \ifnum\value{#1\ArrayCount}>0
        \protected@edef\PoppedValue{\csname #1\roman{#1\ArrayPosition}\endcsname}
        \MakeCommand{#2}{\PoppedValue}
        \addtocounter{#1\ArrayCount}    {-1}
        \addtocounter{#1\ArrayEnd}      {-1}
    \fi
}

\newcommand{\ArrayGet}[2]{% #1 = Array name, % 2 = Array position to get
    \IfGreaterThanEqualTo{#2}{\value{#1\ArrayStart}}{
        \IfGreaterThanEqualTo{\value{#1\ArrayEnd}}{#2}{
            \setcounter{ArrayWorkCounter}{#2}
            \csname #1\roman{ArrayWorkCounter}\endcsname
        }{2}
    }{3}
    
}





\newenvironment{Nomenclature}[1][Nomenclature]{
    
    \newcommand{\NomenclatureTitle}{#1}
    \newlength{\WidestSymbol}
    \newlength{\WidestSymbolTest}
    \setlength{\WidestSymbol}{0pt}
    
    \setlength{\parindent}{0pt}
    
    \ArrayMake{NomenclatureSymbol}
    \ArrayMake{NomenclatureDescription}
    \newcommand{\Add}[2]{
        \ArrayPush{NomenclatureSymbol}{##1}
        \ArrayPush{NomenclatureDescription}{##2}
         
        \settowidth{\WidestSymbolTest}{##1}
        \ifdim\WidestSymbol<\WidestSymbolTest
            \setlength{\WidestSymbol}{\WidestSymbolTest}
        \fi
    }
    \let\item\Add
}{
    \ifnum\value{\NomenclatureSymbolArrayPosition}>1
        \textbf{\NomenclatureTitle}
        \vspace*{0.5em}
        \begin{spacing}{1.5}
            \For[-1]{\value{\NomenclatureSymbolArrayPosition}}{1}{
                \hspace{0.2em}
                \makebox[1.2\WidestSymbol][l]{\ArrayPop{NomenclatureSymbol}}
                \ArrayPop{NomenclatureDescription}
            }
        \end{spacing}
        %\the\WidestSymbol
    \fi
}



\makeatother




